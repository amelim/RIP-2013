\documentclass[12pt]{article}
\usepackage{amsmath}
\usepackage{amssymb,amsmath}
\usepackage{graphics,graphicx,amssymb,verbatim,color}
\usepackage{epstopdf,amsxtra,epsfig, psfrag, subfig} %subfigure
\usepackage{titlesec}
\usepackage[toc,page]{appendix}
\DeclareGraphicsRule{.tif}{png}{.png}{`convert #1 `basename #1 .tif`.png}

%%% Set the margins %%%
\setlength{\topmargin}{-.25in}
\setlength{\headheight}{0in}
\setlength{\headsep}{.5in}
\setlength{\textheight}{8.75in}
\setlength{\textwidth}{7in}
\setlength{\oddsidemargin}{-.125in}
\setlength{\evensidemargin}{0in}

%%% Set paragraph style %%%
\setlength{\parindent}{0pt} 
\setlength{\parskip}{2ex}

%%% Helper Macros %%%
\newcommand{\note}{\textcolor{red}}
\providecommand{\todo}[0]{\textcolor{blue}{TO DO}}
\providecommand{\T}[0]{\textsf{T}}
\providecommand{\ul}[1]{\underline{#1}}
\providecommand{\abs}[1]{\lvert#1\rvert}
\providecommand{\norm}[1]{\left\lVert#1\right\rVert}
\providecommand{\vector}[2]{[#1_1,\ldots,#1_{#2}]} % Vector
\providecommand{\argmin}{\operatornamewithlimits{argmin}} % Minimum argument
\providecommand{\argmax}{\operatornamewithlimits{argmax}} % Maximum argument
\providecommand{\N}{\mathbb{N}} % Natural Numbers
\providecommand{\Z}{\mathbb{Z}} % Integers Numbers
\providecommand{\R}{\mathbb{R}} % Real Numbers
\providecommand{\Im}{\mathcal{R}} % Range Space
\providecommand{\Ker}{\mathcal{N}} % Null Space

\newcommand{\eqnbox}[1]
{
\begin{equation*}
\addtolength{\fboxsep}{1pt}
\boxed{
\begin{split}
#1
\end {split}
}
\end{equation*}
}

%%% Homework Document Macros %%%
% Homework specifics
\newcommand{\hwName}{Assignment 1}
\newcommand{\dueDate}{04/18/2013}
\newcommand{\myFirstName}{Daniel Pickem, Andrew Melim, Jarius Tillman, Matheus Svolenski}
\newcommand{\myLastName}{}
\newcommand{\myIDNumber}{}
\newcommand{\courseNum}{CS 4649/7649}
\newcommand{\courseName}{RIP - Robot Intelligence - Planning}
\newcommand{\courseTerm}{Fall 2013}

% Full Name
\newcommand{\myFullName}{\myFirstName~\myLastName~}

% Document Header - 1st page only
\newcommand{\docHeader}
{
{\myFullName} \\
\noindent\rule[1pt]{\textwidth}{1pt}
\noindent{\courseName ~ \courseNum ~ \courseTerm} \hfill 
\noindent{\hwName} \hfill {\dueDate}
\noindent\rule[8pt]{\textwidth}{1pt}
}

% Running Header - every page
\newcommand{\runningHeader}{\markright{\it{\courseNum ~ \hwName \hfill \myLastName~~~}}}

% New problem
\titleformat{\section}{\large\bfseries}{\thesection}{1em}{}
\newcommand{\problem}[2]{\section*{Problem {#1}~-~{#2}}}
\newcommand{\problemPart}[1]{\subsection*{(#1)}}

% ------------------------------------------------------------------------------------------------------------
% BEGIN DOCUMENT
% ------------------------------------------------------------------------------------------------------------
\begin{document}
\runningHeader
\pagestyle{myheadings}
\thispagestyle{empty}
\docHeader

Deliverables include the following: 
\begin{itemize}
 \item A 4-5 page PDF summary that describes your work, results and answers to the questions posed below. Thoroughness in analysis and answers in the reports are the primary component of your grade!
 \item A page in your summary describing what each group member did to participate in the project.
 \item A zip file that includes your code and results in an organized form.
 \item A README file that describes how to run the code.
 \item Print the summary and bring it to the following class.
\end{itemize}

% ++++++++++++++++++++++++++++++++++++++++++++++++
\problem{1}{Pre-Project: Towers of Hanoi}
\label{sec:problem_1}
% ++++++++++++++++++++++++++++++++++++++++++++++++
\begin{enumerate}
 \item Explain the method by which each of the two planners finds a solution.
 \item Which planner was fastest?
 \item Explain why the winning planner might be more effective on this problem.
\end{enumerate}

% ++++++++++++++++++++++++++++++++++++++++++++++++
\problem{2}{Project Part I: Sokoban PDDL}
\label{sec:problem_2}
% ++++++++++++++++++++++++++++++++++++++++++++++++
\begin{enumerate}
 \item \textbf{Show successful plans from at least one planner on the three Sokoban problems in Figure 2
(1-3). The challenge problem is optional.}
  The plans for problem 2.1 to 2.3 as well as the challenge problem are shown in the appendix.
 \item \textbf{Compare the performance of two planners on this domain. Which one works better? Does this
make sense, why?} \newline
    %
    An overview of the numerical results is given in Table \ref{table:results}. We compared two planners, namely Fast Forward and BlackBox. 

        \begin{itemize}
     \item BlackBox (uses GraphPlan and SAT - see \cite{Blackbox})
     
%      Blackbox is a planning system that works by converting problems specified in STRIPS notation into Boolean satisfiability problems, and then solving the problems with a variety of state-of-the-art satisfiability engines. The front-end employs the graphplan system (Blum and Furst 1995). There is extreme flexibility in specifying the engines to use. For example, you can tell it to use walksat (Selman, Kautz, and Cohen 1994) for 60 seconds, and if that fails, then satz (Li and Anbulagan 1997) for 1000 seconds. This gives blackbox the capability of functioning efficiently over a broad range of problems. The name blackbox refers to the fact that the plan generator knows nothing about the SAT solvers, and the SAT solvers know nothing about plans: each is a "black box" to the other.
     \item Fast Forward (see \cite{FastForward})
     
%      FF is a forward chaining heuristic state space planner. The main heuristic principle was originally developed by Blai Bonet and Hector Geffner for the HSP system: to obtain a heuristic estimate, relax the task P at hand into a simpler task P+ by ignoring the delete lists of all operators. While HSP employs a technique that gives a rough estimate for the solution length of P+ , FF extracts an explicit solution to P+ , by using a GRAPHPLAN-style algorithm. The number of actions in the relaxed solutions is used as a goal distance estimate. These estimates control a novel local search strategy, enforced hill-climbing: this is a hill-climbing procedure that, in each intermediate state, uses breadth first search to find a strictly better, possibly indirect, successor. As a second important heuristic information, the relaxed plans can be used to prune the search space: usually, the actions that are really useful in a state are contained in the relaxed plan, so one can restrict the successors of any state to those produced by members of the respective relaxed solution. FF employs a slightly more elaborated form of this heuristic, which we call helpful actions pruning. The simple architecture described so far already solves most of the available benchmarks extremely efficiently. Problematic cases are when there are dead ends --- states from which the goals get unreachable --- or goal orderings. In the presence of the latter phenomenon, like in the Blocksworld, the local search sometimes proceeds too greedily, and gets trapped. To overcome this, we have integrated the Goal Agenda algorithm (first proposed by Jana Koehler), as well as a simple goal ordering technique of our own, based on the relaxed solutions. In order to deal with dead end states, which can cause the search to fail entirely, we have chosen a simple safety-net solution: if local search fails, then we skip everything done so far and switch to a complete best-first algorithm that simply expands all search nodes by increasing order of goal distance evaluation.
    \end{itemize}
    
    \begin{table}%[t!]
    \label{table:results}
    \caption{Performance Comparison of BlackBox and FF.}
    \begin{tabular*}{1.0\textwidth}{c|c|c|c}
    \hline
    \bfseries Problem Instance & \bfseries FF (default) & \bfseries BlackBox (default) & \bfseries BlackBox (walksat) \\
    \hline\hline
    % DATA
    % ########################################################
    Problem 2.1 [time] 		&  0 	& 0 	& 0 \\ \hline
    Problem 2.1 [steps]		&  0 	& 0 	& 0 \\ \hline\hline
    Problem 2.2 [time] 		&  0	& 0 	& 0 \\ \hline
    Problem 2.2 [steps]		&  0 	& 0 	& 0 \\ \hline\hline
    Problem 2.3 [time] 		&  0 	& 0 	& 0 \\ \hline
    Problem 2.3 [steps]		&  0 	& 0 	& 0 \\ \hline\hline
    Problem Challenge [time] 	&  0 	& 0 	& 0 \\ \hline
    Problem Challenge [steps]	&  0 	& 0 	& 0 \\ \hline
    % ########################################################
    \hline
    \end{tabular*}
    \end{table}

 \item \textbf{Clearly PDDL was not intended for this sort of application. Discuss the challenges in expressing geometric constraints in semantic planning.}
  \begin{itemize}
   \item PDDL can't handle continuous worlds, the world has to be discretized.
   \item There is no inherent notion of adjacency or neighborhood. Such relations have to be expressed explicitly. E.g. if squares are used, for each square PDDL requires 4 adjacency clauses.
   \item In PDDL each square has to be given a label instead of just saying if the robot moves one step to the right, its x-coordinate increased by one. This leads to a large amount of labels and adjacency relationships in the problem description.
   \item It needs to be explicitly stated whether a square is empty or contains a box or robot in order to detect an obstacle. Each action taken must ensure to update the 'empty' property of a square. This is a classical manifestation of the frame problem.
   \item Even for small problem instances as the ones shown in the examples it was too tedious to write down the PDDL instantiation by hand. What could be described easily as a matrix required roughly 100 lines of PDDL code. 
  \end{itemize}

 \item \textbf{In many cases, geometric and dynamic planning are insufficient to describe a domain. Give
an example of a problem that is best suited for semantic (classical) planning. Explain why a
semantic representation would be desirable.}
  \begin{itemize}
   \item Clearly, geometric constraints can't be efficiently expressed in PDDL. Therefore, all problem types that have a physical interpretation and an embodiment in a physical and geometric world are not well suited for PDDL. Domains with geometric constraints that encode for example positions of objects or distances between objects (in continuous or discrete space) are best handled by geometric planners. 
   \item Problems that represent an abstract world, however, where neither states nor actions are tied to a geometric interpretation of the world are well suited for PDDL and classical semantic planning. In these types of problems dimensions/size of objects do not matter and actions are not tied to geometric constraints. In this case constraints define the abstract rules of the domain rather than a geometric setup. 
   \item A very classic problem in semantic planning is for example the 'change tire problem' (see \cite{Russell2010}) that describes how to change a flat tire on a car. The \textit{move} action moves objects from one abstract location to another (e.g. from the trunk to an axle). These locations hold no geometric information and are essentially just labels for objects in the world. The dimensions of objects do not matter either. Such a domain where the problem has been abstracted away from the geometric constraints are well-suited for a semantic description such as PDDL.
  \end{itemize}

\end{enumerate}

% ++++++++++++++++++++++++++++++++++++++++++++++++
\problem{3}{Project Part II: Sokoban Planner}
\label{sec:problem_3}
% ++++++++++++++++++++++++++++++++++++++++++++++++
\begin{enumerate}
  \item Give successful plans from your planner on the Sokoban problems in Figure 2 and any others.
  \item Compare the performance of your planner to the PDDL planners you used in the previous
problem. Which was faster? Why?
  \item Prove that your planner was complete. Your instructor has a math background: a proof ”is
a convincing argument.” Make sure you address each aspect of completeness and why your
planner satisfies it. Pictures are always welcome.
  \item What methods did you use to speed up the planning? Give a short description of each method and explain why it did or didn't help on each relevant problem.
\end{enumerate}

% ++++++++++++++++++++++++++++++++++++++++++++++++
\problem{4}{Post-Project: Towers of Hanoi Revisited}
\label{sec:problem_4}
% ++++++++++++++++++++++++++++++++++++++++++++++++
\begin{enumerate}
 \item Give successful plans from at least one planner with 6 and 10 disks.
 \item Do you notice anything about the structure of the plans? Can you use this to increase the
efficiency of planning for Towers of Hanoi? Explain.
 \item In a paragraph or two, explain a general planning strategy that would take advantage of
problem structure. Make sure your strategy applies to problems other than Towers of Hanoi.
Would such a planner still be complete?
\end{enumerate}

\newpage
\begin{center}
\Huge{Summary} 
\end{center}
This section summarizes the contribution of each team member. 

\begin{enumerate}
  \item Matheus Svolenski
  \item Jarius Tillman
  \item Andrew Melim
  \item Daniel Pickem
    \begin{itemize}
     \item Implemented the class structure of the planner from problem 3.
     \item Implemented the world class for the planner that describes and stores the environment and Sokoban map.
     \item Implemented the A* algorithm.
     \item Created PDDL description for problem 2.
     \item Ran BlackBox and FF planners on the Towers of Hanoi problem 1 and problem 4.
     \item Wrote Matlab tool to autogenerate Tower of Hanoi PDDL description.
     \item Wrote Matlab tool to autogenerate Sokoban PDDL description.
    \end{itemize}
\end{enumerate}

% ##########################################
% 		Bibliography
% ##########################################
\begin{thebibliography}{50}
  %\bibitem[label]{key}.
  \bibitem{Russell2010} Russell, Stuart J. and Norvig, Peter, \textsl{Artificial Intelligence: A Modern Approach}, Pearson Education, 2010
  \bibitem{Blackbox} BlackBox Planner, \textsl{http://www.cs.rochester.edu/~kautz/satplan/blackbox/\#super}
  \bibitem{FastForward} Fast Forward Planner, \textsl{http://fai.cs.uni-saarland.de/hoffmann/ff.html}
\end{thebibliography}

\newpage
\begin{appendix}
\problem{2}{Project Part I: Sokoban PDDL - Successful Plans}
  \begin{itemize}
    \item Problem 2.1:
      \begin{enumerate}
	\item: MOVE R S-3-3 S-3-4 UP
	\item: MOVE R S-3-4 S-4-4 RIGHT
	\item: MOVE R S-4-4 S-5-4 RIGHT
	\item: MOVE R S-5-4 S-5-5 UP
	\item: PUSH R B1 S-5-5 S-4-5 S-3-5 LEFT
	\item: MOVE R S-4-5 S-4-4 DOWN
	\item: MOVE R S-4-4 S-3-4 LEFT
	\item: MOVE R S-3-4 S-2-4 LEFT
	\item: MOVE R S-2-4 S-2-5 UP
	\item: MOVE R S-2-5 S-2-6 UP
	\item : MOVE R S-2-6 S-3-6 RIGHT
	\item: PUSH R B1 S-3-6 S-3-5 S-3-4 DOWN
	\item: PUSH R B1 S-3-5 S-3-4 S-3-3 DOWN
	\item: PUSH R B1 S-3-4 S-3-3 S-3-2 DOWN
      \end{enumerate}
    \item Problem 2.2:
      \begin{enumerate}
	\item: MOVE R S-2-3 S-2-2 DOWN
	\item: MOVE R S-2-2 S-3-2 RIGHT
	\item: MOVE R S-3-2 S-4-2 RIGHT
	\item: PUSH R B2 S-4-2 S-4-3 S-4-4 UP
	\item: MOVE R S-4-3 S-4-2 DOWN
	\item: MOVE R S-4-2 S-3-2 LEFT
	\item: MOVE R S-3-2 S-2-2 LEFT
	\item: MOVE R S-2-2 S-2-3 UP
	\item: PUSH R B1 S-2-3 S-3-3 S-4-3 RIGHT
	\item: MOVE R S-3-3 S-3-4 UP
	\item: PUSH R B2 S-3-4 S-4-4 S-5-4 RIGHT
	\item: MOVE R S-4-4 S-3-4 LEFT
	\item: MOVE R S-3-4 S-3-3 DOWN
	\item: MOVE R S-3-3 S-3-2 DOWN
	\item: MOVE R S-3-2 S-4-2 RIGHT
	\item: PUSH R B1 S-4-2 S-4-3 S-4-4 UP
	\item: MOVE R S-4-3 S-5-3 RIGHT
	\item: PUSH R B2 S-5-3 S-5-4 S-5-5 UP
	\item: PUSH R B2 S-5-4 S-5-5 S-5-6 UP
	\item: PUSH R B2 S-5-5 S-5-6 S-5-7 UP
	\item: MOVE R S-5-6 S-5-5 DOWN
	\item: MOVE R S-5-5 S-4-5 LEFT
	\item: PUSH R B1 S-4-5 S-4-4 S-4-3 DOWN
	\item: MOVE R S-4-4 S-5-4 RIGHT
	\item: MOVE R S-5-4 S-5-3 DOWN
	\item: PUSH R B1 S-5-3 S-4-3 S-3-3 LEFT
	\item: MOVE R S-4-3 S-4-4 UP
	\item: MOVE R S-4-4 S-3-4 LEFT
	\item: PUSH R B1 S-3-4 S-3-3 S-3-2 DOWN
	\item: MOVE R S-3-3 S-4-3 RIGHT
	\item: MOVE R S-4-3 S-4-2 DOWN
	\item: PUSH R B1 S-4-2 S-3-2 S-2-2 LEFT
      \end{enumerate}
  \item Problem 2.3
	0: MOVE R S-3-5 S-4-5 RIGHT
        1: MOVE R S-4-5 S-4-6 UP
        2: MOVE R S-4-6 S-4-7 UP
        3: MOVE R S-4-7 S-5-7 RIGHT
        4: MOVE R S-5-7 S-6-7 RIGHT
        5: MOVE R S-6-7 S-6-6 DOWN
        6: MOVE R S-6-6 S-6-5 DOWN
        7: PUSH R B1 S-6-5 S-5-5 S-4-5 LEFT
        8: PUSH R B1 S-5-5 S-4-5 S-3-5 LEFT
        9: MOVE R S-4-5 S-5-5 RIGHT
       10: MOVE R S-5-5 S-6-5 RIGHT
       11: MOVE R S-6-5 S-6-6 UP
       12: MOVE R S-6-6 S-6-7 UP
       13: MOVE R S-6-7 S-7-7 RIGHT
       14: MOVE R S-7-7 S-8-7 RIGHT
       15: MOVE R S-8-7 S-8-6 DOWN
       16: MOVE R S-8-6 S-8-5 DOWN
       17: MOVE R S-8-5 S-8-4 DOWN
       18: MOVE R S-8-4 S-8-3 DOWN
       19: MOVE R S-8-3 S-8-2 DOWN
       20: MOVE R S-8-2 S-9-2 RIGHT
       21: MOVE R S-9-2 S-10-2 RIGHT
       22: MOVE R S-10-2 S-10-3 UP
       23: MOVE R S-10-3 S-10-4 UP
       24: MOVE R S-10-4 S-10-5 UP
       25: PUSH R B3 S-10-5 S-9-5 S-8-5 LEFT
       26: MOVE R S-9-5 S-10-5 RIGHT
       27: MOVE R S-10-5 S-10-4 DOWN
       28: MOVE R S-10-4 S-10-3 DOWN
       29: MOVE R S-10-3 S-10-2 DOWN
       30: MOVE R S-10-2 S-9-2 LEFT
       31: MOVE R S-9-2 S-8-2 LEFT
       32: MOVE R S-8-2 S-8-3 UP
       33: MOVE R S-8-3 S-8-4 UP
       34: PUSH R B3 S-8-4 S-8-5 S-8-6 UP
       35: PUSH R B2 S-8-5 S-7-5 S-6-5 LEFT
       36: PUSH R B2 S-7-5 S-6-5 S-5-5 LEFT
       37: MOVE R S-6-5 S-6-6 UP
       38: MOVE R S-6-6 S-6-7 UP
       39: MOVE R S-6-7 S-5-7 LEFT
       40: MOVE R S-5-7 S-4-7 LEFT
       41: MOVE R S-4-7 S-4-6 DOWN
       42: MOVE R S-4-6 S-4-5 DOWN
       43: PUSH R B2 S-4-5 S-5-5 S-6-5 RIGHT
       44: PUSH R B2 S-5-5 S-6-5 S-7-5 RIGHT
       45: MOVE R S-6-5 S-5-5 LEFT
       46: MOVE R S-5-5 S-4-5 LEFT
       47: MOVE R S-4-5 S-4-6 UP
       48: MOVE R S-4-6 S-4-7 UP
       49: MOVE R S-4-7 S-3-7 LEFT
       50: MOVE R S-3-7 S-2-7 LEFT
       51: MOVE R S-2-7 S-2-6 DOWN
       52: MOVE R S-2-6 S-2-5 DOWN
       53: PUSH R B1 S-2-5 S-3-5 S-4-5 RIGHT
       54: PUSH R B1 S-3-5 S-4-5 S-5-5 RIGHT
       55: MOVE R S-4-5 S-4-6 UP
       56: MOVE R S-4-6 S-4-7 UP
       57: MOVE R S-4-7 S-5-7 RIGHT
       58: MOVE R S-5-7 S-6-7 RIGHT
       59: MOVE R S-6-7 S-6-6 DOWN
       60: MOVE R S-6-6 S-6-5 DOWN
       61: PUSH R B1 S-6-5 S-5-5 S-4-5 LEFT
       62: MOVE R S-5-5 S-6-5 RIGHT
       63: MOVE R S-6-5 S-6-6 UP
       64: MOVE R S-6-6 S-6-7 UP
       65: MOVE R S-6-7 S-7-7 RIGHT
       66: MOVE R S-7-7 S-8-7 RIGHT
       67: PUSH R B3 S-8-7 S-8-6 S-8-5 DOWN
       68: PUSH R B3 S-8-6 S-8-5 S-8-4 DOWN
       69: PUSH R B3 S-8-5 S-8-4 S-8-3 DOWN
       70: PUSH R B3 S-8-4 S-8-3 S-8-2 DOWN
       71: MOVE R S-8-3 S-8-4 UP
       72: MOVE R S-8-4 S-8-5 UP
       73: MOVE R S-8-5 S-8-6 UP
       74: MOVE R S-8-6 S-8-7 UP
       75: MOVE R S-8-7 S-7-7 LEFT
       76: MOVE R S-7-7 S-6-7 LEFT
       77: MOVE R S-6-7 S-6-6 DOWN
       78: MOVE R S-6-6 S-6-5 DOWN
       79: PUSH R B2 S-6-5 S-7-5 S-8-5 RIGHT
       80: MOVE R S-7-5 S-6-5 LEFT
       81: MOVE R S-6-5 S-6-6 UP
       82: MOVE R S-6-6 S-6-7 UP
       83: MOVE R S-6-7 S-7-7 RIGHT
       84: MOVE R S-7-7 S-8-7 RIGHT
       85: MOVE R S-8-7 S-8-6 DOWN
       86: PUSH R B2 S-8-6 S-8-5 S-8-4 DOWN
       87: PUSH R B2 S-8-5 S-8-4 S-8-3 DOWN
       88: MOVE R S-8-4 S-8-5 UP
       89: MOVE R S-8-5 S-7-5 LEFT
       90: MOVE R S-7-5 S-6-5 LEFT
       91: MOVE R S-6-5 S-6-6 UP
       92: MOVE R S-6-6 S-6-7 UP
       93: MOVE R S-6-7 S-5-7 LEFT
       94: MOVE R S-5-7 S-4-7 LEFT
       95: MOVE R S-4-7 S-3-7 LEFT
       96: MOVE R S-3-7 S-2-7 LEFT
       97: MOVE R S-2-7 S-2-6 DOWN
       98: MOVE R S-2-6 S-2-5 DOWN
       99: MOVE R S-2-5 S-3-5 RIGHT
      100: PUSH R B1 S-3-5 S-4-5 S-5-5 RIGHT
      101: PUSH R B1 S-4-5 S-5-5 S-6-5 RIGHT
      102: PUSH R B1 S-5-5 S-6-5 S-7-5 RIGHT
      103: PUSH R B1 S-6-5 S-7-5 S-8-5 RIGHT
      104: MOVE R S-7-5 S-6-5 LEFT
      105: MOVE R S-6-5 S-6-6 UP
      106: MOVE R S-6-6 S-6-7 UP
      107: MOVE R S-6-7 S-7-7 RIGHT
      108: MOVE R S-7-7 S-8-7 RIGHT
      109: MOVE R S-8-7 S-8-6 DOWN
      110: PUSH R B1 S-8-6 S-8-5 S-8-4 DOWN
    \item Problem 2.4 Challenge
     0: MOVE R S-6-4 S-6-5 UP
        1: MOVE R S-6-5 S-5-5 LEFT
        2: MOVE R S-5-5 S-4-5 LEFT
        3: MOVE R S-4-5 S-3-5 LEFT
        4: MOVE R S-3-5 S-2-5 LEFT
        5: MOVE R S-2-5 S-2-4 DOWN
        6: PUSH R B1 S-2-4 S-3-4 S-4-4 RIGHT
        7: PUSH R B1 S-3-4 S-4-4 S-5-4 RIGHT
        8: MOVE R S-4-4 S-4-5 UP
        9: MOVE R S-4-5 S-4-6 UP
       10: MOVE R S-4-6 S-4-7 UP
       11: MOVE R S-4-7 S-3-7 LEFT
       12: PUSH R B2 S-3-7 S-3-6 S-3-5 DOWN
       13: MOVE R S-3-6 S-4-6 RIGHT
       14: MOVE R S-4-6 S-4-5 DOWN
       15: MOVE R S-4-5 S-4-4 DOWN
       16: MOVE R S-4-4 S-3-4 LEFT
       17: MOVE R S-3-4 S-2-4 LEFT
       18: MOVE R S-2-4 S-2-5 UP
       19: PUSH R B2 S-2-5 S-3-5 S-4-5 RIGHT
       20: MOVE R S-3-5 S-3-6 UP
       21: MOVE R S-3-6 S-4-6 RIGHT
       22: MOVE R S-4-6 S-5-6 RIGHT
       23: MOVE R S-5-6 S-5-5 DOWN
       24: MOVE R S-5-5 S-6-5 RIGHT
       25: MOVE R S-6-5 S-6-4 DOWN
       26: PUSH R B1 S-6-4 S-5-4 S-4-4 LEFT
       27: PUSH R B1 S-5-4 S-4-4 S-3-4 LEFT
       28: MOVE R S-4-4 S-5-4 RIGHT
       29: MOVE R S-5-4 S-5-5 UP
       30: MOVE R S-5-5 S-5-6 UP
       31: MOVE R S-5-6 S-4-6 LEFT
       32: MOVE R S-4-6 S-3-6 LEFT
       33: MOVE R S-3-6 S-3-5 DOWN
       34: PUSH R B2 S-3-5 S-4-5 S-5-5 RIGHT
       35: MOVE R S-4-5 S-4-4 DOWN
       36: MOVE R S-4-4 S-5-4 RIGHT
       37: MOVE R S-5-4 S-5-3 DOWN
       38: MOVE R S-5-3 S-5-2 DOWN
       39: MOVE R S-5-2 S-4-2 LEFT
       40: PUSH R B3 S-4-2 S-4-3 S-4-4 UP
       41: PUSH R B3 S-4-3 S-4-4 S-4-5 UP
       42: PUSH R B3 S-4-4 S-4-5 S-4-6 UP
       43: MOVE R S-4-5 S-3-5 LEFT
       44: MOVE R S-3-5 S-2-5 LEFT
       45: MOVE R S-2-5 S-2-4 DOWN
       46: PUSH R B1 S-2-4 S-3-4 S-4-4 RIGHT
       47: MOVE R S-3-4 S-3-5 UP
       48: MOVE R S-3-5 S-4-5 RIGHT
       49: PUSH R B1 S-4-5 S-4-4 S-4-3 DOWN
       50: MOVE R S-4-4 S-5-4 RIGHT
       51: MOVE R S-5-4 S-6-4 RIGHT
       52: MOVE R S-6-4 S-6-5 UP
       53: PUSH R B2 S-6-5 S-5-5 S-4-5 LEFT
       54: PUSH R B2 S-5-5 S-4-5 S-3-5 LEFT
       55: MOVE R S-4-5 S-5-5 RIGHT
       56: MOVE R S-5-5 S-5-6 UP
       57: PUSH R B3 S-5-6 S-4-6 S-3-6 LEFT
       58: MOVE R S-4-6 S-4-5 DOWN
       59: MOVE R S-4-5 S-4-4 DOWN
       60: MOVE R S-4-4 S-3-4 LEFT
       61: MOVE R S-3-4 S-2-4 LEFT
       62: MOVE R S-2-4 S-2-5 UP
       63: PUSH R B2 S-2-5 S-3-5 S-4-5 RIGHT
       64: PUSH R B2 S-3-5 S-4-5 S-5-5 RIGHT
       65: MOVE R S-4-5 S-4-6 UP
       66: MOVE R S-4-6 S-4-7 UP
       67: MOVE R S-4-7 S-3-7 LEFT
       68: PUSH R B3 S-3-7 S-3-6 S-3-5 DOWN
       69: MOVE R S-3-6 S-4-6 RIGHT
       70: MOVE R S-4-6 S-4-5 DOWN
       71: MOVE R S-4-5 S-4-4 DOWN
       72: MOVE R S-4-4 S-5-4 RIGHT
       73: MOVE R S-5-4 S-5-3 DOWN
       74: MOVE R S-5-3 S-5-2 DOWN
       75: MOVE R S-5-2 S-4-2 LEFT
       76: PUSH R B1 S-4-2 S-4-3 S-4-4 UP
       77: PUSH R B1 S-4-3 S-4-4 S-4-5 UP
  \end{itemize}

\end{appendix}

\newpage
\end{document}