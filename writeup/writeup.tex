\documentclass[12pt]{article}
\usepackage{amsmath}
\usepackage{amssymb,amsmath}
\usepackage{graphics,graphicx,amssymb,verbatim,color}
\usepackage{epstopdf,amsxtra,epsfig, psfrag, subfig} %subfigure
\usepackage{titlesec}
\DeclareGraphicsRule{.tif}{png}{.png}{`convert #1 `basename #1 .tif`.png}

%%% Set the margins %%%
\setlength{\topmargin}{-.25in}
\setlength{\headheight}{0in}
\setlength{\headsep}{.5in}
\setlength{\textheight}{8.75in}
\setlength{\textwidth}{7in}
\setlength{\oddsidemargin}{-.125in}
\setlength{\evensidemargin}{0in}

%%% Set paragraph style %%%
\setlength{\parindent}{0pt} 
\setlength{\parskip}{2ex}

%%% Helper Macros %%%
\newcommand{\note}{\textcolor{red}}
\providecommand{\todo}[0]{\textcolor{blue}{TO DO}}
\providecommand{\T}[0]{\textsf{T}}
\providecommand{\ul}[1]{\underline{#1}}
\providecommand{\abs}[1]{\lvert#1\rvert}
\providecommand{\norm}[1]{\left\lVert#1\right\rVert}
\providecommand{\vector}[2]{[#1_1,\ldots,#1_{#2}]} % Vector
\providecommand{\argmin}{\operatornamewithlimits{argmin}} % Minimum argument
\providecommand{\argmax}{\operatornamewithlimits{argmax}} % Maximum argument
\providecommand{\N}{\mathbb{N}} % Natural Numbers
\providecommand{\Z}{\mathbb{Z}} % Integers Numbers
\providecommand{\R}{\mathbb{R}} % Real Numbers
\providecommand{\Im}{\mathcal{R}} % Range Space
\providecommand{\Ker}{\mathcal{N}} % Null Space

\newcommand{\eqnbox}[1]
{
\begin{equation*}
\addtolength{\fboxsep}{1pt}
\boxed{
\begin{split}
#1
\end {split}
}
\end{equation*}
}

%%% Homework Document Macros %%%
% Homework specifics
\newcommand{\hwName}{Assignment 1}
\newcommand{\dueDate}{04/18/2013}
\newcommand{\myFirstName}{Daniel Pickem, Andrew Melim, Jarius Tillman, Matheus Svolenski}
\newcommand{\myLastName}{}
\newcommand{\myIDNumber}{}
\newcommand{\courseNum}{CS 4649/7649}
\newcommand{\courseName}{RIP - Robot Intelligence - Planning}
\newcommand{\courseTerm}{Fall 2013}

% Full Name
\newcommand{\myFullName}{\myFirstName~\myLastName~}

% Document Header - 1st page only
\newcommand{\docHeader}
{
{\myFullName} \\
\noindent\rule[1pt]{\textwidth}{1pt}
\noindent{\courseName ~ \courseNum ~ \courseTerm} \hfill 
\noindent{\hwName} \hfill {\dueDate}
\noindent\rule[8pt]{\textwidth}{1pt}
}

% Running Header - every page
\newcommand{\runningHeader}{\markright{\it{\courseNum ~ \hwName \hfill \myLastName~~~}}}

% New problem
\titleformat{\section}{\large\bfseries}{\thesection}{1em}{}
\newcommand{\problem}[2]{\section*{Problem {#1}~-~{#2}}}
\newcommand{\problemPart}[1]{\subsection*{(#1)}}

% ------------------------------------------------------------------------------------------------------------
% BEGIN DOCUMENT
% ------------------------------------------------------------------------------------------------------------
\begin{document}
\runningHeader
\pagestyle{myheadings}
\thispagestyle{empty}
\docHeader

Deliverables include the following: 
\begin{itemize}
 \item A 4-5 page PDF summary that describes your work, results and answers to the questions posed below. Thoroughness in analysis and answers in the reports are the primary component of your grade!
 \item A page in your summary describing what each group member did to participate in the project.
 \item A zip file that includes your code and results in an organized form.
 \item A README file that describes how to run the code.
 \item Print the summary and bring it to the following class.
\end{itemize}

% ++++++++++++++++++++++++++++++++++++++++++++++++
\problem{1}{Pre-Project: Towers of Hanoi}
% ++++++++++++++++++++++++++++++++++++++++++++++++
\begin{enumerate}
 \item Explain the method by which each of the two planners finds a solution.
 \item Which planner was fastest?
 \item Explain why the winning planner might be more effective on this problem.
\end{enumerate}

% ++++++++++++++++++++++++++++++++++++++++++++++++
\problem{2}{Project Part I: Sokoban PDDL}
% ++++++++++++++++++++++++++++++++++++++++++++++++
\begin{enumerate}
 \item Show successful plans from at least one planner on the three Sokoban problems in Figure 2
(1-3). The challenge problem is optional.
 \item Compare the performance of two planners on this domain. Which one works better? Does this
make sense, why?
 \item Clearly PDDL was not intended for this sort of application. Discuss the challenges in expressing geometric constraints in semantic planning.
 \item In many cases, geometric and dynamic planning are insufficient to describe a domain. Give
an example of a problem that is best suited for semantic (classical) planning. Explain why a
semantic representation would be desirable.
\end{enumerate}

% ++++++++++++++++++++++++++++++++++++++++++++++++
\problem{3}{Project Part II: Sokoban Planner}
% ++++++++++++++++++++++++++++++++++++++++++++++++
\begin{enumerate}
  \item Give successful plans from your planner on the Sokoban problems in Figure 2 and any others.
  \item Compare the performance of your planner to the PDDL planners you used in the previous
problem. Which was faster? Why?
  \item Prove that your planner was complete. Your instructor has a math background: a proof ”is
a convincing argument.” Make sure you address each aspect of completeness and why your
planner satisfies it. Pictures are always welcome.
  \item What methods did you use to speed up the planning? Give a short description of each method and explain why it did or didn't help on each relevant problem.
\end{enumerate}

% ++++++++++++++++++++++++++++++++++++++++++++++++
\problem{4}{Post-Project: Towers of Hanoi Revisited}
% ++++++++++++++++++++++++++++++++++++++++++++++++
\begin{enumerate}
 \item Give successful plans from at least one planner with 6 and 10 disks.
 \item Do you notice anything about the structure of the plans? Can you use this to increase the
efficiency of planning for Towers of Hanoi? Explain.
 \item In a paragraph or two, explain a general planning strategy that would take advantage of
problem structure. Make sure your strategy applies to problems other than Towers of Hanoi.
Would such a planner still be complete?
\end{enumerate}


\newpage
\end{document}