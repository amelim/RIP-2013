\documentclass[12pt]{article}
\usepackage{amsmath}
\usepackage{amssymb,amsmath}
\usepackage{graphics,graphicx,amssymb,verbatim,color}
\usepackage{epstopdf,amsxtra,epsfig, psfrag, subfig} %subfigure
\usepackage{titlesec}
\usepackage[toc,page]{appendix}
\DeclareGraphicsRule{.tif}{png}{.png}{`convert #1 `basename #1 .tif`.png}

%%% Set the margins %%%
\setlength{\topmargin}{-.25in}
\setlength{\headheight}{0in}
\setlength{\headsep}{.5in}
\setlength{\textheight}{8.75in}
\setlength{\textwidth}{7in}
\setlength{\oddsidemargin}{-.125in}
\setlength{\evensidemargin}{0in}

%%% Set paragraph style %%%
\setlength{\parindent}{0pt} 
\setlength{\parskip}{2ex}

%%% Helper Macros %%%
\newcommand{\note}{\textcolor{red}}
\providecommand{\todo}[0]{\textcolor{blue}{TO DO}}
\providecommand{\T}[0]{\textsf{T}}
\providecommand{\ul}[1]{\underline{#1}}
\providecommand{\abs}[1]{\lvert#1\rvert}
\providecommand{\norm}[1]{\left\lVert#1\right\rVert}
\providecommand{\vector}[2]{[#1_1,\ldots,#1_{#2}]} % Vector
\providecommand{\argmin}{\operatornamewithlimits{argmin}} % Minimum argument
\providecommand{\argmax}{\operatornamewithlimits{argmax}} % Maximum argument
\providecommand{\N}{\mathbb{N}} % Natural Numbers
\providecommand{\Z}{\mathbb{Z}} % Integers Numbers
\providecommand{\R}{\mathbb{R}} % Real Numbers
\providecommand{\Im}{\mathcal{R}} % Range Space
\providecommand{\Ker}{\mathcal{N}} % Null Space

\newcommand{\eqnbox}[1]
{
\begin{equation*}
\addtolength{\fboxsep}{1pt}
\boxed{
\begin{split}
#1
\end {split}
}
\end{equation*}
}

%%% Homework Document Macros %%%
% Homework specifics
\newcommand{\hwName}{Assignment 1}
\newcommand{\dueDate}{04/18/2013}
\newcommand{\myFirstName}{Daniel Pickem, Andrew Melim, Jarius Tillman, Matheus Svolenski}
\newcommand{\myLastName}{}
\newcommand{\myIDNumber}{}
\newcommand{\courseNum}{CS 4649/7649}
\newcommand{\courseName}{RIP - Robot Intelligence - Planning}
\newcommand{\courseTerm}{Fall 2013}

% Full Name
\newcommand{\myFullName}{\myFirstName~\myLastName~}

% Document Header - 1st page only
\newcommand{\docHeader}
{
{\myFullName} \\
\noindent\rule[1pt]{\textwidth}{1pt}
\noindent{\courseName ~ \courseNum ~ \courseTerm} \hfill 
\noindent{\hwName} \hfill {\dueDate}
\noindent\rule[8pt]{\textwidth}{1pt}
}

% Running Header - every page
\newcommand{\runningHeader}{\markright{\it{\courseNum ~ \hwName \hfill \myLastName~~~}}}

% New problem
\titleformat{\section}{\large\bfseries}{\thesection}{1em}{}
\newcommand{\problem}[2]{\section*{Problem {#1}~-~{#2}}}
\newcommand{\problemPart}[1]{\subsection*{(#1)}}

% ------------------------------------------------------------------------------------------------------------
% BEGIN DOCUMENT
% ------------------------------------------------------------------------------------------------------------
\begin{document}
\runningHeader
\pagestyle{myheadings}
\thispagestyle{empty}
\docHeader

Deliverables include the following: 
\begin{itemize}
 \item A 4-5 page PDF summary that describes your work, results and answers to the questions posed below. Thoroughness in analysis and answers in the reports are the primary component of your grade!
 \item A page in your summary describing what each group member did to participate in the project.
 \item A zip file that includes your code and results in an organized form.
 \item A README file that describes how to run the code.
 \item Print the summary and bring it to the following class.
\end{itemize}

% ++++++++++++++++++++++++++++++++++++++++++++++++
\problem{1}{Pre-Project: Towers of Hanoi}
\label{sec:problem_1}
% ++++++++++++++++++++++++++++++++++++++++++++++++
\begin{enumerate}
 \item  \textbf{Explain the method by which each of the two planners finds a solution.} \newline
To test the Towers of Hanoi problem, we compared two diferent planners, named BlackBox and Fast Forward (FF).
 \begin{itemize}
      \item \textbf{BlackBox (see \cite{Blackbox})}: BlackBox is a planning system that works by converting problems specified in STRIPS notation into Boolean satisfiability problems. It employs a graph plan system, later converted, that is solved by any of a variety of fast SAT engines.
      \item \textbf{Fast Forward (see \cite{FastForward})}: FF relies in a forward search in the state space, guided by a heuristic that estimates goal distances by ignoring delete lists. It uses an enforced hill-climbing search, that searches for the direct successors of a path, until finding a value that is better than the one it started with. It performs a complete breadth first search (BFS) for a state with strictly better evaluation. However, this method depends on the structure of the problem. If a local search fails, FF will skip everything done so far and switch to a complete best-first algorithm that simply expands all search nodes by increasing order of goal distance evaluation. This was the case in the Towers of Hanoi problem.
 \end{itemize}
 \item  \textbf{Which planner was fastest?} \newline
The fastest planner was the Fast Forward (FF). Its performance can be analysed through \ref{table:results1}, that shows how long it took for each solution to be found and in how many steps this happend. In most cases, BlackBox was not able to find a solution, giving up after 50 iterations. The outputs are all included in the .zip file that accompains the project.

   \begin{table}%[t!]
    \label{table:results1}
    \caption{Performance Comparison of BlackBox and FF (timing results in seconds).}
    \begin{tabular*}{1.0\textwidth}{c|c|c|c|c}
    \hline
    \bfseries Problem Instance & \bfseries Fast Forward & \bfseries BlackBox \\
    \hline\hline
    % DATA
    % ########################################################
    Hanoi 3 Disks [time] 		&  0.00 	& 0.01	\\ \hline    
    Hanoi 6 Disks [time] 		&  0.01	& -	\\ \hline
    Hanoi 6 (other team) Disks [time] 		&  0.01 	& -	\\ \hline
    Hanoi 10 Disks [time] 	&  0.89  & -	\\ \hline\hline
    Hanoi 3 Disks [steps]		&  7 	& 7 	\\ \hline
    Hanoi 6 Disks [steps]		&  63  	& - 	\\ \hline
    Hanoi 6 (other team) Disks [steps]		&  63 	& - 	\\ \hline
    Hanoi 10 Disks [steps]	&  1023 	& - 	\\ \hline
    % ########################################################
    \hline
    \end{tabular*}
    \end{table}

 \item  \textbf{Explain why the winning planner might be more effective on this problem.} \newline
\end{enumerate}

% ++++++++++++++++++++++++++++++++++++++++++++++++
\problem{2}{Project Part I: Sokoban PDDL}
\label{sec:problem_2}
% ++++++++++++++++++++++++++++++++++++++++++++++++
\begin{enumerate}
 \item \textbf{Show successful plans from at least one planner on the three Sokoban problems in Figure 2
(1-3). The challenge problem is optional.}
  The plans for problem 2.1 to 2.3 as well as the challenge problem are shown in the appendix.
 \item \textbf{Compare the performance of two planners on this domain. Which one works better? Does this
make sense, why?} \newline
    %
    An overview of the numerical results is given in Table \ref{table:results}. We compared three planners, namely Fast Forward (FF), Conformant-FF, and BlackBox. FF and Conformant-FF seem to use a very similar underlying architecture while BlackBox relies on a different representation of the problem as a satisfiability problem.
    
    \begin{itemize}
      \item \textbf{BlackBox (see \cite{Blackbox})}: BlackBox converts the PDDL description of the problem into a Boolean satisfiability problem and solves it using a variety of satisfiability engines. The front-end of BlackBox relies on the Graphplan system.
      \item \textbf{Fast Forward (see \cite{FastForward})}: FF is a forward chaining heuristic state space planner. The heuristic relaxes the task at hand into a simpler task by ignoring the delete list of all operators. That relaxed problem is solved using a GraphPlan-style algorithm and the solution to the relaxed problem is used as heuristic to guide the search. The relaxed plans are also used to prune the search space.
%      FF is a forward chaining heuristic state space planner. The main heuristic principle was originally developed by Blai Bonet and Hector Geffner for the HSP system: to obtain a heuristic estimate, relax the task P at hand into a simpler task P+ by ignoring the delete lists of all operators. While HSP employs a technique that gives a rough estimate for the solution length of P+ , FF extracts an explicit solution to P+ , by using a GRAPHPLAN-style algorithm. The number of actions in the relaxed solutions is used as a goal distance estimate. These estimates control a novel local search strategy, enforced hill-climbing: this is a hill-climbing procedure that, in each intermediate state, uses breadth first search to find a strictly better, possibly indirect, successor. As a second important heuristic information, the relaxed plans can be used to prune the search space: usually, the actions that are really useful in a state are contained in the relaxed plan, so one can restrict the successors of any state to those produced by members of the respective relaxed solution. FF employs a slightly more elaborated form of this heuristic, which we call helpful actions pruning. The simple architecture described so far already solves most of the available benchmarks extremely efficiently. Problematic cases are when there are dead ends --- states from which the goals get unreachable --- or goal orderings. In the presence of the latter phenomenon, like in the Blocksworld, the local search sometimes proceeds too greedily, and gets trapped. To overcome this, we have integrated the Goal Agenda algorithm (first proposed by Jana Koehler), as well as a simple goal ordering technique of our own, based on the relaxed solutions. In order to deal with dead end states, which can cause the search to fail entirely, we have chosen a simple safety-net solution: if local search fails, then we skip everything done so far and switch to a complete best-first algorithm that simply expands all search nodes by increasing order of goal distance evaluation.
    \end{itemize}
    
    As expected (see Table \ref{table:results}), FF and Conformant-FF achieve much better results and generally solve the problems in much shorter times. BlackBox takes orders of magnitude longer and fails to compute plans for problems that take more than 25 steps to solve (i.e. the solver times out). The number of steps BlackBox can handle seems to be fairly low. The first up to 20 levels of the graph are computed fairly fast, but any level in the GraphPlan graph after that take on the order of seconds or minutes. Clearly the number of levels is too low to solve a Sokoban level like the ones shown in problem 2.2 to 2.4. \\
    It seems that, unlike FF, BlackBox does not use any heuristic to prune the search tree or search in a more goal-directed fashion. FF prunes large parts of the search space using the heuristic described above, which allows FF to handle much larger problem instances and compute a plan much faster. 
    
    \begin{table}%[t!]
    \label{table:results}
    \caption{Performance Comparison of BlackBox, FF, and Conformant-FF (timing results in seconds).}
    \begin{tabular*}{1.0\textwidth}{c|c|c|c|c}
    \hline
    \bfseries Problem Instance & \bfseries FF (default) & \bfseries Conformant-FF & \bfseries BlackBox (default) \\
    \hline\hline
    % DATA
    % ########################################################
    Problem 2.1 [time] 		&  0.0 	& 0.0 	& 0.27 	\\ \hline    
    Problem 2.2 [time] 		&  0.0	& 0.02 	& -	\\ \hline
    Problem 2.3 [time] 		&  0.02 & 0.12 	& -	\\ \hline
    Problem Challenge [time] 	&  0.7 	& 34.54 & -	\\ \hline\hline
    Problem 2.1 [steps]		&  12 	& 15 	& 14 	\\ \hline
    Problem 2.2 [steps]		&  31 	& 31 	& - 	\\ \hline
    Problem 2.3 [steps]		&  110 	& 88 	& - 	\\ \hline
    Problem Challenge [steps]	&  77 	& 77 	& - 	\\ \hline
    % ########################################################
    \hline
    \end{tabular*}
    \end{table}

 \item \textbf{Clearly PDDL was not intended for this sort of application. Discuss the challenges in expressing geometric constraints in semantic planning.}
  \begin{itemize}
   \item PDDL can't handle continuous worlds, the world has to be discretized.
   \item There is no inherent notion of adjacency or neighborhood. Such relations have to be expressed explicitly. E.g. if squares are used, for each square PDDL requires 4 adjacency clauses.
   \item In PDDL each square has to be given a label instead of just saying if the robot moves one step to the right, its x-coordinate increased by one. This leads to a large amount of labels and adjacency relationships in the problem description.
   \item It needs to be explicitly stated whether a square is empty or contains a box or robot in order to detect an obstacle. Each action taken must ensure to update the 'empty' property of a square. This is a classical manifestation of the frame problem.
   \item Even for small problem instances as the ones shown in the examples it was too tedious to write down the PDDL instantiation by hand. What could be described easily as a matrix required roughly 100 lines of PDDL code. 
  \end{itemize}

 \item \textbf{In many cases, geometric and dynamic planning are insufficient to describe a domain. Give
an example of a problem that is best suited for semantic (classical) planning. Explain why a
semantic representation would be desirable.}
  \begin{itemize}
   \item Clearly, geometric constraints can't be efficiently expressed in PDDL. Therefore, all problem types that have a physical interpretation and an embodiment in a physical and geometric world are not well suited for PDDL. Domains with geometric constraints that encode for example positions of objects or distances between objects (in continuous or discrete space) are best handled by geometric planners. 
   \item Problems that represent an abstract world, however, where neither states nor actions are tied to a geometric interpretation of the world are well suited for PDDL and classical semantic planning. In these types of problems dimensions/size of objects do not matter and actions are not tied to geometric constraints. In this case constraints define the abstract rules of the domain rather than a geometric setup. 
   \item A very classic problem in semantic planning is for example the 'change tire problem' (see \cite{Russell2010}) that describes how to change a flat tire on a car. The \textit{move} action moves objects from one abstract location to another (e.g. from the trunk to an axle). These locations hold no geometric information and are essentially just labels for objects in the world. The dimensions of objects do not matter either. Such a domain where the problem has been abstracted away from the geometric constraints are well-suited for a semantic description such as PDDL.
  \end{itemize}

\end{enumerate}

% ++++++++++++++++++++++++++++++++++++++++++++++++
\problem{3}{Project Part II: Sokoban Planner}
\label{sec:problem_3}
% ++++++++++++++++++++++++++++++++++++++++++++++++
\begin{enumerate}
  \item \textbf{Give successful plans from your planner on the Sokoban problems in Figure 2 and any others.} 
  See the attached file \textbf{sokoban astar sols} for the solutions.
 
  \item \textbf{Compare the performance of your planner to the PDDL planners you used in the previous
problem. Which was faster? Why?} \\
Our results (see Table \ref{table:results2}) shows two different cost functions for our A* implementation. The column on the left shows utlizes the cost of the actions already taken by the robot in addition to the Manhattan distance of boxes from their starting location. The column on the right compares with just using the box movement cost. \\\\
The results are quite illuminating in that we can see the trade off between efficiency and plan quality. With the action cost included, while the code can solve the simple problems fairly efficiently, the time to find a solution grows quite significantly as the difficult of the problem grows. This approach does actually find high quality solutions on par with FF and Conformant-FF. \\\\
On the other hand, we found significant speed up in computation time when ignoring the robot action cost with the trade off of longer plans. This makes conceptual sense as it's easy to realize that a method which ignores robot movement could take extra steps to reduce box movement.  Furthermore, the expansion of the search space is reduced since the algorithm runs the box quickly to the goal, ignoring better possible paths. However, we still were not able to match FF's combination of efficiency and solution quality.

    \begin{table}%[t!]
    \label{table:results2}
    \caption{Performance results of our A* algorithm (timing results in seconds).}
    \begin{tabular*}{1.0\textwidth}{c|c|c}
    \hline
    \bfseries Problem Instance & \bfseries A* with action cost  & \bfseries A* without action cost \\
    \hline\hline
    % DATA
    % ########################################################
    Problem 2.1 [time] 		&  0.01 seconds &  0.0 seconds 	\\ \hline    
    Problem 2.2 [time] 		&  0.16 seconds &  0.08 seconds 	\\ \hline
    Problem 2.3 [time] 		&  3.24 seconds &  0.62 seconds 	\\ \hline
    Problem Challenge [time] 	&  411.14 seconds &  5.45 seconds 	\\ \hline\hline
    Problem 2.1 [steps]		&  14 steps &  14 steps 	\\ \hline
    Problem 2.2 [steps]		&  32 steps &  36 steps 	\\ \hline
    Problem 2.3 [steps]		&  93 steps &  102 steps 	\\ \hline
    Problem Challenge [steps]	&  76 steps &  120 steps 	\\ \hline
    % ########################################################
    \hline
    \end{tabular*}
    \end{table}
  \item \textbf{Prove that your planner was complete. Your instructor has a math background: a proof ”is
a convincing argument.” Make sure you address each aspect of completeness and why your
planner satisfies it. Pictures are always welcome.}\\
Our A* implementation keeps track of an open and closed set of expanded states. Each iteration we expand all possible states from the current state with the lowest cost from the open set. For all possible expansions we check to see if the same state configuration already exists in the closed set (that is, the set of states already expanded), if not it's added to the open set with it's cost. If an expanded state results in a goal condition, the algorithm returns the state and all actions taken to get to that state.\\
The process of searching the open set will check all feasible state configurations. The closed set guarantees that no state will be expanded multiple times. Therefore, assuming a bounded world which will not expand off into infinity, once the open set is exhausted the algorithm will terminate. This shows that the algorithm is complete since it is guaranteed to exhauste the open set or find a solution given a bounded world.

  \item \textbf{What methods did you use to speed up the planning? Give a short description of each method and explain why it did or didn't help on each relevant problem.}\\
  As mentioned above, the use of different cost functions found a trade off between efficiency and solution quality. A bi-directional A* search, where two simultaneous searchs start from both the initial state and a goal state, would improve the planning speed. It's possible that a bi-directional A* with a cost function including action costs could improve the efficiency while maintaining the better solutions found in our slower results.
\end{enumerate}

% ++++++++++++++++++++++++++++++++++++++++++++++++
\problem{4}{Post-Project: Towers of Hanoi Revisited}
\label{sec:problem_4}
% ++++++++++++++++++++++++++++++++++++++++++++++++
\begin{enumerate}
 \item \textbf{Give successful plans from at least one planner with 6 and 10 disks.} \newline The solution for both the 6 disk and 10 disk problem were solved using FF. The 6 disk solution took 0.00 seconds to compute and was solved in 63 steps. The 10 disk solution took 0.36 seconds to compute and was solved in 1023 steps. The 6 disk solution is provided in the appendix. Both the 6 disk and 10 disk solution are attached in the .zip submission
 \item \textbf{Do you notice anything about the structure of the plans? Can you use this to increase the
efficiency of planning for Towers of Hanoi? Explain.} \newline
   The plans were very repetitive. For instance, the plan for 10 disks contains in some form a similar plan for 6 disks multiple times throughout the entire solution. This is because in order to move the next largest disk off of the pole, another pole needs to be completely empty. This scenario requires all smaller sized disks to be stacked on the third pole in the correct order. The most efficient way for this to occur is to reuse the solution already created. In general, in order to improve the logic of the plan it would be efficient to group these steps into sub plans that can be reexecuted quickly. There would need to be some reasoning to determine which pole to move to first, but after that the rest of the movements are predetermined. 
 \item \textbf{In a paragraph or two, explain a general planning strategy that would take advantage of
problem structure. Make sure your strategy applies to problems other than Towers of Hanoi.
Would such a planner still be complete?}
+\newline One possible strategy would be to create a planner that could self expand. For instance, it could state with a single disk tower of hanoi, moving from pole x to pole z. Then solve a 2 disk game moving a stacked tower on pole x to a stacked tower on pole z. This program would continually iterate and hopefully through some form of machine learning detect repeatedly grouped steps. For instance, the optimal steps for moving a 2 disk stack from pole x to pole y: move-stack(disk2,pole-x,pole-y) can be simplified to 3 substeps: move(disk1,pole.x,pole.z), move (disk2,pole.x,pole.y),move(disk1,pole.z,pole.y). The next move-stack action \- move-stack (disk3,pole.x,pole.y) would simply involve \- move-stack (disk2,pole.x,pole.z), \- move (disk3,pole.x,pole.y), \- move-stack (disk2,pole.z,pole.y). Hopefully this program would be able to detect other progressively complex tasks that require similar recursion methods.
\end{enumerate}

\newpage
\begin{center}
\Huge{Summary} 
\end{center}
This section summarizes the contribution of each team member. 

\begin{enumerate}
  \item Matheus Svolenski
    \begin{itemize}
     \item Implemented the parser class to properly extract the Sokoban world information.
     \item Ran BlackBox and FF planners on the Towers of Hanoi problems.
    \end{itemize}

  \item Jarius Tillman
  \item Andrew Melim
    \begin{itemize}
     \item Wrote project build system
     \item Implemented state class and expansion of states for Sokoban domain.
     \item Code debugging and verification via unit testing for A* solver
     \item Created PDDL description for problem 2
    \end{itemize}

  \item Daniel Pickem
    \begin{itemize}
     \item Implemented the class structure of the planner from problem 3.
     \item Implemented the world class for the planner that describes and stores the environment and Sokoban map.
     \item Implemented the A* algorithm.
     \item Created PDDL description for problem 2.
     \item Ran BlackBox and FF planners on the Towers of Hanoi problem 1 and problem 4.
     \item Wrote Matlab tool to autogenerate Tower of Hanoi PDDL description.
     \item Wrote Matlab tool to autogenerate Sokoban PDDL description.
    \end{itemize}
\end{enumerate}

% ##########################################
% 		Bibliography
% ##########################################
\begin{thebibliography}{50}
  %\bibitem[label]{key}.
  \bibitem{Russell2010} Russell, Stuart J. and Norvig, Peter, \textsl{Artificial Intelligence: A Modern Approach}, Pearson Education, 2010
  \bibitem{Blackbox} BlackBox Planner, \textsl{http://www.cs.rochester.edu/~kautz/satplan/blackbox/\#super}
  \bibitem{FastForward} Fast Forward Planner, \textsl{http://fai.cs.uni-saarland.de/hoffmann/ff.html}
\end{thebibliography}

\newpage
\begin{appendix}
\problem{2}{Project Part I: Sokoban PDDL - Successful Plans}
  All generated plans can be found in the attached .zip file, specifically in the results folder. The relatively short plans for problem 2.1 and problem 2.2 are shown below. 
  \begin{itemize}
    \item Problem 2.1:
      \begin{enumerate}
	\item: MOVE R S-3-3 S-3-4 UP
	\item: MOVE R S-3-4 S-4-4 RIGHT
	\item: MOVE R S-4-4 S-5-4 RIGHT
	\item: MOVE R S-5-4 S-5-5 UP
	\item: PUSH R B1 S-5-5 S-4-5 S-3-5 LEFT
	\item: MOVE R S-4-5 S-4-4 DOWN
	\item: MOVE R S-4-4 S-3-4 LEFT
	\item: MOVE R S-3-4 S-2-4 LEFT
	\item: MOVE R S-2-4 S-2-5 UP
	\item: MOVE R S-2-5 S-2-6 UP
	\item : MOVE R S-2-6 S-3-6 RIGHT
	\item: PUSH R B1 S-3-6 S-3-5 S-3-4 DOWN
	\item: PUSH R B1 S-3-5 S-3-4 S-3-3 DOWN
	\item: PUSH R B1 S-3-4 S-3-3 S-3-2 DOWN
      \end{enumerate}
    \item Problem 2.2:
      \begin{enumerate}
	\item: MOVE R S-2-3 S-2-2 DOWN
	\item: MOVE R S-2-2 S-3-2 RIGHT
	\item: MOVE R S-3-2 S-4-2 RIGHT
	\item: PUSH R B2 S-4-2 S-4-3 S-4-4 UP
	\item: MOVE R S-4-3 S-4-2 DOWN
	\item: MOVE R S-4-2 S-3-2 LEFT
	\item: MOVE R S-3-2 S-2-2 LEFT
	\item: MOVE R S-2-2 S-2-3 UP
	\item: PUSH R B1 S-2-3 S-3-3 S-4-3 RIGHT
	\item: MOVE R S-3-3 S-3-4 UP
	\item: PUSH R B2 S-3-4 S-4-4 S-5-4 RIGHT
	\item: MOVE R S-4-4 S-3-4 LEFT
	\item: MOVE R S-3-4 S-3-3 DOWN
	\item: MOVE R S-3-3 S-3-2 DOWN
	\item: MOVE R S-3-2 S-4-2 RIGHT
	\item: PUSH R B1 S-4-2 S-4-3 S-4-4 UP
	\item: MOVE R S-4-3 S-5-3 RIGHT
	\item: PUSH R B2 S-5-3 S-5-4 S-5-5 UP
	\item: PUSH R B2 S-5-4 S-5-5 S-5-6 UP
	\item: PUSH R B2 S-5-5 S-5-6 S-5-7 UP
	\item: MOVE R S-5-6 S-5-5 DOWN
	\item: MOVE R S-5-5 S-4-5 LEFT
	\item: PUSH R B1 S-4-5 S-4-4 S-4-3 DOWN
	\item: MOVE R S-4-4 S-5-4 RIGHT
	\item: MOVE R S-5-4 S-5-3 DOWN
	\item: PUSH R B1 S-5-3 S-4-3 S-3-3 LEFT
	\item: MOVE R S-4-3 S-4-4 UP
	\item: MOVE R S-4-4 S-3-4 LEFT
	\item: PUSH R B1 S-3-4 S-3-3 S-3-2 DOWN
	\item: MOVE R S-3-3 S-4-3 RIGHT
	\item: MOVE R S-4-3 S-4-2 DOWN
	\item: PUSH R B1 S-4-2 S-3-2 S-2-2 LEFT
      \end{enumerate}
    \item Problem 2.3: See attached .zip file.
    \item Problem 2.4 Challenge: See attached .zip file.
  \end{itemize}
\problem{4}{Post-Project: Towers of Hanoi Revisited}
  \begin{itemize}
    \item Problem 4.1.a: 6 Disks Using FF
	\begin{enumerate}
	\item  MOVE-DISK D1 D2 P2
	\item  MOVE-DISK D2 D3 P1
	\item  MOVE-DISK D1 P2 D2
	\item  MOVE-DISK D3 D4 P2
	\item  MOVE-DISK D1 D2 D4
	\item  MOVE-DISK D2 P1 D3
	\item  MOVE-DISK D1 D4 D2
	\item  MOVE-DISK D4 D5 P1
	\item  MOVE-DISK D1 D2 D4
	\item  MOVE-DISK D2 D3 D5
	\item  MOVE-DISK D1 D4 D2
	\item  MOVE-DISK D3 P2 D4
	\item  MOVE-DISK D1 D2 P2
	\item  MOVE-DISK D2 D5 D3
	\item  MOVE-DISK D1 P2 D2
	\item  MOVE-DISK D5 D6 P2
	\item  MOVE-DISK D1 D2 D6
	\item  MOVE-DISK D2 D3 D5
	\item  MOVE-DISK D1 D6 D2
	\item  MOVE-DISK D3 D4 D6
	\item  MOVE-DISK D1 D2 D4
	\item  MOVE-DISK D2 D5 D3
	\item  MOVE-DISK D1 D4 D2
	\item  MOVE-DISK D4 P1 D5
	\item  MOVE-DISK D1 D2 D4
	\item  MOVE-DISK D2 D3 P1
	\item  MOVE-DISK D1 D4 D2
	\item  MOVE-DISK D3 D6 D4
	\item  MOVE-DISK D1 D2 D6
	\item  MOVE-DISK D2 P1 D3
	\item  MOVE-DISK D1 D6 D2
	\item  MOVE-DISK D6 P3 P1
	\item  MOVE-DISK D1 D2 D6
	\item  MOVE-DISK D2 D3 P3
	\item  MOVE-DISK D1 D6 D2
	\item  MOVE-DISK D3 D4 D6
	\item  MOVE-DISK D1 D2 D4
	\item  MOVE-DISK D2 P3 D3
	\item  MOVE-DISK D1 D4 D2
	\item  MOVE-DISK D4 D5 P3
	\item  MOVE-DISK D1 D2 D4
	\item  MOVE-DISK D2 D3 D5
	\item  MOVE-DISK D1 D4 D2
	\item  MOVE-DISK D3 D6 D4
	\item  MOVE-DISK D1 D2 D6
	\item  MOVE-DISK D2 D5 D3
	\item  MOVE-DISK D1 D6 D2
	\item  MOVE-DISK D5 P2 D6
	\item  MOVE-DISK D1 D2 P2
	\item  MOVE-DISK D2 D3 D5
	\item  MOVE-DISK D1 P2 D2
	\item  MOVE-DISK D3 D4 P2
	\item  MOVE-DISK D1 D2 D4
	\item  MOVE-DISK D2 D5 D3
	\item  MOVE-DISK D1 D4 D2
	\item  MOVE-DISK D4 P3 D5
	\item  MOVE-DISK D1 D2 D4
	\item  MOVE-DISK D2 D3 P3
	\item  MOVE-DISK D1 D4 D2
	\item  MOVE-DISK D3 P2 D4
	\item  MOVE-DISK D1 D2 P2
	\item  MOVE-DISK D2 P3 D3
	\item  MOVE-DISK D1 P2 D2
      \end{enumerate}
   
  \end{itemize}

\end{appendix}

\newpage
\end{document}
